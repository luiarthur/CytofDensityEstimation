\documentclass[12pt]{article} % 12-point font

\usepackage[margin=1in]{geometry} % set page to 1-inch margins
\usepackage{bm,bbm} % for math
\usepackage{amsmath} % for math
\usepackage{amssymb} % like \Rightarrow
\setlength\parindent{0pt} % Suppresses the indentation of new paragraphs.

% Big display
\newcommand{\ds}{ \displaystyle }
% Parenthesis
\newcommand{\norm}[1]{\left\lVert#1\right\rVert}
\newcommand{\p}[1]{\left(#1\right)}
\newcommand{\bk}[1]{\left[#1\right]}
\newcommand{\bc}[1]{ \left\{#1\right\} }
\newcommand{\abs}[1]{ \left|#1\right| }
% Derivatives
\newcommand{\df}[2]{ \frac{d#1}{d#2} }
\newcommand{\ddf}[2]{ \frac{d^2#1}{d{#2}^2} }
\newcommand{\pd}[2]{ \frac{\partial#1}{\partial#2} }
\newcommand{\pdd}[2]{\frac{\partial^2#1}{\partial{#2}^2} }
% Distributions
\newcommand{\Normal}{\text{Normal}}
\newcommand{\Beta}{\text{Beta}}
\newcommand{\G}{\text{Gamma}}
\newcommand{\InvGamma}{\text{Inv-Gamma}}
\newcommand{\Uniform}{\text{Uniform}}
\newcommand{\Dirichlet}{\text{Dirichlet}}
\newcommand{\LogNormal}{\text{LogNormal}}
% Statistics
\newcommand{\E}{ \text{E} }
\newcommand{\iid}{\overset{iid}{\sim}}
\newcommand{\ind}{\overset{ind}{\sim}}
\newcommand{\true}{\text{TRUE}}

\usepackage{color}
\newcommand{\alert}[1]{\textcolor{red}{#1}}


% Graphics
\usepackage{graphicx}  % for figures
\usepackage{float} % Put figure exactly where I want [H]

% Uncomment if using bibliography
% Bibliography
% \usepackage{natbib}
% \bibliographystyle{plainnat}

% Adds settings for hyperlinks. (Mainly for table of contents.)
\usepackage{hyperref}
\hypersetup{
  pdfborder={0 0 0} % removes red box from links
}

% Title Settings
\title{Simulation Study 1}
\author{Arthur Lui}
\date{\today} % \date{} to set date to empty

% MAIN %
\begin{document}
\maketitle

\section{Simulation Setup}\label{sec:sim-setup}
We assessed our model through the following simulation studies. We first
generated four data sets (I, II, III, IV) according to our model. In each
four scenarios,
%
the true mixture locations were $\bm{\mu}^\true=(-1, 1, 3)$,
the true mixture scales were $\bm{\sigma}^\true=(0.7, 0.7, 0.7)$,
the true mixture degrees of freedom were $\bm{\nu}^\true=(7, 5, 10)$, and
the true mixture skews were $\bm{\phi}^\true=(-5, -3, 0)$.
%
In scenario I, $\gamma_C^\true=0.3$, $\gamma_T^\true=0.2$, $\bm\eta_C^\true=(0.5,
0.5, 0)$, and $\bm\eta_T^\true=(0.5,0.4,0.1)$. Implicitly, $\beta^\true=1$.
In scenario II, $\gamma_C^\true=0.3$, $\gamma_T^\true=0.3$, $\bm\eta_C^\true=(0.5,
0.5, 0)$, and $\bm\eta_T^\true=(0.5,0.4,0.1)$. Implicitly, $\beta^\true=1$.
In scenario III, $\gamma_C^\true=0.3$, $\gamma_T^\true=0.3$, $\bm\eta_C^\true=(0.5,
0.5, 0)$, and $\bm\eta_T^\true=(0.5,0.45,0.05)$. Implicitly, $\beta^\true=1$.
In scenario IV, $\gamma_C^\true=0.3$, $\gamma_T^\true=0.3$, $\bm\eta_C^\true=(0.5,
0.5, 0)$, and $\bm\eta_T^\true=(0.5,0.5,0)$. Implicitly, $\beta^\true=0$.
%
In each scenario, $N_i=1000$. Table~\ref{tab:sim-truth} summarizes the
simulation truth for the model parameters.
\begin{table}
  \centering
  \begin{tabular}{|c|ccccccc|}
    \hline 
    Scenario & $\gamma_C^\true$ & $\gamma_T^\true$ & $\bm\eta_C^\true$ & 
    $\bm\eta_T^\true$ & $\beta^\true$ & $\hat\beta$ & KS p-value \\
    \hline 
    I   & 0.3 & 0.2 & (0.5, 0.5, 0) & (0.5, 0.40, 0.10) & 1 & 1.000 & $0.000091$ \\
    II  & 0.3 & 0.3 & (0.5, 0.5, 0) & (0.5, 0.40, 0.10) & 1 & 1.000 & $0.001724$ \\
    III & 0.3 & 0.3 & (0.5, 0.5, 0) & (0.5, 0.45, 0.05) & 1 & 0.991 & $0.148338$ \\
    IV  & 0.3 & 0.3 & (0.5, 0.5, 0) & (0.5, 0.50, 0.00) & 1 & 0.002 & $0.500360$ \\
    \hline
  \end{tabular}
  \caption{Simulation truth under various scenarios. Posterior mean of
  $\beta$ is included before the right-most column. The right-most column
  is the p-value of under the two-sample Kolmogorov-Smirnov test.}
  \label{tab:sim-truth}
\end{table}

\section{Simulation Results}\label{sec:sim-results}
The following priors were used in this analysis. First, we set $K=3$ and
$p=0.5$. Then $\gamma_i\sim\Beta(1, 1)$, $\bm\eta_i\sim\Dirichlet_K(1/K)$,
$\mu_k\sim\Normal(\bar{\mu}, s_\mu^2)$, $\omega_k\sim\InvGamma(0.1, 0.1)$,
$\nu_k\sim\LogNormal(1.6, 0.4)$, $\psi_k\sim\Normal(-1, 1)$, where,
respectively, $\bar{\mu}$ (-0.557) and $s_\mu$ (1.276) are the empirical mean
and standard deviation of the data for which $y_{i,n} > 0$. Posterior
inference was made via Gibbs sampling. The initial 2000 MCMC samples were
discarded as burn-in, and the subsequent 10000 samples were kept for
posterior inference. In addition, updating the parameters $\bm\zeta, \bm
v, \bm\mu, \bm\omega, \bm\nu$, and $\bm\psi$ multiple times for each
update of the other parameters helped with mixing. Thus, 10 updates for
those parameters were done during each iteration of the MCMC. 
\alert{Juhee: Are multiple updates for other parameters needed when $\beta$
switches states?}
The inference speed was approximately 54 iterations per second.
Figure~\ref{fig:sim-postdens-data-kde} summarizes the posterior densities for
the positive values of $y_{i,n}$. The dashed lines are kernel density
estimates of the data, and the shaded regions are the 95\% credible intervals
for the densities. Note that the intervals match the data closely.
% Also, in scenario III, where $\beta^\true=0$, the posterior mean of $\beta$
% was also 0. Thus, $\gamma_T$ and $\bm\eta_T$ were simply samples from the
% prior. Thus, the posterior density for sample T was not included here.

\alert{Juhee: I found a few dumb typos in the code. After I corrected them,
things look much better. With a large enough sample size, the parameters can
be recovered reasonably well. The recovery isn't perfect as parameters like
the skewness and degrees of freedom are still hard to estimate, but they are
quite close. It appears that with more data, they are better recovered.}

\begin{figure}[t!]
  \centering
  \begin{tabular}{cc}
    \includegraphics[scale=.45]{results/scenario1/img/postpred.pdf} &
    \includegraphics[scale=.45]{results/scenario2/img/postpred.pdf} \\
    (a) Scenario I &
    (b) Scenario II \\
    \includegraphics[scale=.45]{results/scenario3/img/postpred.pdf} &
    \includegraphics[scale=.45]{results/scenario4/img/postpred.pdf} \\
    (c) Scenario III &
    (d) Scenario IV
  \end{tabular}
  \caption{Posterior density in each simulation scenario for observed data
  ($y_{i,n}>0$). Dashed lines are the kernel density estimates of the
  simulated data. The shaded regions are 95\% credible intervals of the
  density.}
  \label{fig:sim-postdens-data-kde}
\end{figure}

Since the KDE is only an approximation for the density of the observed data,
we have included Figure~\ref{fig:sim-postdens-data-true-den}, which replaces
the KDE of the observed portion of the simulated data with the actual pdf of
the data-generating mechanism. The graphs more clearly show that the simulation
truth is well captured by this model. 

\begin{figure}[t!]
  \centering
  \begin{tabular}{cc}
    \includegraphics[scale=.45]{results/scenario1/img/postpred-true-data-density.pdf} &
    \includegraphics[scale=.45]{results/scenario2/img/postpred-true-data-density.pdf} \\
    (a) Scenario I &
    (b) Scenario II \\
    \includegraphics[scale=.45]{results/scenario3/img/postpred-true-data-density.pdf} &
    \includegraphics[scale=.45]{results/scenario4/img/postpred-true-data-density.pdf} \\
    (c) Scenario III &
    (d) Scenario IV
  \end{tabular}
  \caption{Posterior density in each simulation scenario for observed data
  ($y_{i,n}>0$). Dashed lines are the kernel density estimates of the
  simulated data. The shaded regions are 95\% credible intervals of the
  density.}
  \label{fig:sim-postdens-data-true-den}
\end{figure}

Figure~\ref{fig:sim-gammas} shows box plots of the posterior distribution of
$\gamma_i$ for the different scenarios. The circles represent the proportion
of zeros in the simulated data. The posterior distributions easily capture the
true values of $\gamma_i$. Note that in Scenario IV, since $\beta=0$ with high probability,
$\gamma_T$ is simply being sampled from the prior.
\begin{figure}[t!]
  \centering
  \begin{tabular}{cc}
    (a) Scenario I &
    (b) Scenario II \\
    \includegraphics[scale=.45]{results/scenario1/img/gammas.pdf} &
    \includegraphics[scale=.45]{results/scenario2/img/gammas.pdf} \\
    (c) Scenario III &
    (d) Scenario IV \\
    \includegraphics[scale=.45]{results/scenario3/img/gammas.pdf} &
    \includegraphics[scale=.45]{results/scenario4/img/gammas.pdf}
  \end{tabular}
  \caption{Box plots of posterior distribution of $\gamma_i$ for different
  simulated datasets. Circles represent the proportion of zeros in each
  dataset. Blue for sample C, and red for sample T.}
  \label{fig:sim-gammas}
\end{figure}


% Uncomment if using bibliography:
% \bibliography{bib}
\end{document}