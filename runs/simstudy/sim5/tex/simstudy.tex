\documentclass[12pt]{article} % 12-point font

\usepackage[margin=1in]{geometry} % set page to 1-inch margins
\usepackage{bm,bbm} % for math
\usepackage{amsmath} % for math
\usepackage{amssymb} % like \Rightarrow
\setlength\parindent{0pt} % Suppresses the indentation of new paragraphs.

% Big display
\newcommand{\ds}{ \displaystyle }
% Parenthesis
\newcommand{\norm}[1]{\left\lVert#1\right\rVert}
\newcommand{\p}[1]{\left(#1\right)}
\newcommand{\bk}[1]{\left[#1\right]}
\newcommand{\bc}[1]{ \left\{#1\right\} }
\newcommand{\abs}[1]{ \left|#1\right| }
% Derivatives
\newcommand{\df}[2]{ \frac{d#1}{d#2} }
\newcommand{\ddf}[2]{ \frac{d^2#1}{d{#2}^2} }
\newcommand{\pd}[2]{ \frac{\partial#1}{\partial#2} }
\newcommand{\pdd}[2]{\frac{\partial^2#1}{\partial{#2}^2} }
% Distributions
\newcommand{\Normal}{\text{Normal}}
\newcommand{\Beta}{\text{Beta}}
\newcommand{\G}{\text{Gamma}}
\newcommand{\InvGamma}{\text{Inv-Gamma}}
\newcommand{\Uniform}{\text{Uniform}}
\newcommand{\Dirichlet}{\text{Dirichlet}}
\newcommand{\LogNormal}{\text{LogNormal}}
% Statistics
\newcommand{\E}{ \text{E} }
\newcommand{\iid}{\overset{iid}{\sim}}
\newcommand{\ind}{\overset{ind}{\sim}}
\newcommand{\true}{\text{TRUE}}

\usepackage{color}
\newcommand{\alert}[1]{\textcolor{red}{#1}}


% Graphics
\usepackage{graphicx}  % for figures
\usepackage{float} % Put figure exactly where I want [H]

% Uncomment if using bibliography
% Bibliography
% \usepackage{natbib}
% \bibliographystyle{plainnat}

% Adds settings for hyperlinks. (Mainly for table of contents.)
\usepackage{hyperref}
\hypersetup{
  pdfborder={0 0 0} % removes red box from links
}

% Title Settings
\title{Simulation Study 5}
\author{Arthur Lui}
\date{\today} % \date{} to set date to empty

% MAIN %
\begin{document}
\maketitle

\section{Simulation Setup}\label{sec:sim-setup}
We assessed our model through the following simulation studies. We first
generated four data sets (I, II, III, IV) according to our model.
Table~\ref{tab:sim-truth} contains the simulation truth of the model
parameters under the four scenarios. Figure~\ref{fig:sim-truth-density}
contains the histogram of the observed data and the true density of the data
in each scenario. The figures also list the observed proportion of zeros in
the datasets. Note that scenario I and II, the densities of the simulation
truths are visibly different, while the true proportions of zeros are the
same in both samples. In scenario III, the densities of the two samples are
the same, but the true proportions of zeros differ. In scenario IV, the
densities and proportions of zeros are identical in the simulation truth. In
each scenario, $N_i=1000$, and $K^\true=4$. These scenarios were created to 
imitate real data we will analyze.
\begin{table}
  \centering
  \begin{tabular}{|c|cccc|}
    \hline 
    & Scenario I & Scenario II & Scenario III & Scenario IV \\
    \hline 
    $\beta$     & 1 & 1 & 1 & 0 \\
    $\gamma_C$  & 0.1 & 0.1 & 0.1 & 0.1 \\
    $\gamma_T$  & 0.1 & 0.1 & 0.05 & 0.1 \\
    $\bm\eta_C$ & (0.25,0.75,0,0) & (0.05,0.05,0.5,0.4) & (0.05,0.05,0.5,0.4) & (0.05,0.05,0.5,0.4) \\
    $\bm\eta_T$ & (0.05,0.05,0.9,0.5) & (0.05,0.05,0.9,0) & (0.05,0.05,0.5,0.4) & (0.05,0.05,0.5,0.4) \\
    $\bm\mu$    & (-1.5,3.5,5.1,5) & (-1.5,3.5,1.5,4.3) & (-1.5,3.5,5.1,4.3) & (-1.5,3.5,5.1,4.3) \\
    $\bm\sigma$ & (1.6,1.76,1.76,1.6) & (1.6,1.76,1.76,1.6) & (1.6,1.76,1.76,1.6) & (1.6,1.76,1.76,1.6) \\
    $\bm\nu$    & (12,10,10,15) & (12,10,10,15) & (12,10,10,15) & (12,10,10,15) \\
    $\bm\phi$   & (0,-10,-10,0) & (12,10,10,-11) & (0,-10,-10,-11) & (0,-10,-10,-11) \\
    \hline
  \end{tabular}
  \caption{Simulation truth of model parameters under the four scenarios.}
  \label{tab:sim-truth}
\end{table}

\begin{figure}[t!]
  \centering
  \begin{tabular}{cc}
    (a) Scenario I & (b) Scenario II \\
    \includegraphics[scale=0.5]{results/K6/scenario1/m0/img/simdata.pdf} &
    \includegraphics[scale=0.5]{results/K6/scenario2/m0/img/simdata.pdf} \\
    (c) Scenario III & (d) Scenario IV \\
    \includegraphics[scale=0.5]{results/K6/scenario3/m0/img/simdata.pdf} &
    \includegraphics[scale=0.5]{results/K6/scenario4/m0/img/simdata.pdf}
  \end{tabular}
  \caption{Histogram of simulated data and simulation truth density.}
  \label{fig:sim-truth-density}
\end{figure}
%
\begin{table}
  \centering
  \begin{tabular}{|c|crccc|}
    \hline 
    Scenario & $\beta^\true$ & Bayes Factor & $\Pr(\beta=1)$ &
    $\Pr(\beta=1\mid \bm y)$ & KS p-value \\
    \hline 
    I   & 1 & 437.0691 & 0.5 & $\approx 1$ & $<10^{-6}$ \\
    II  & 1 &  14.8797 & 0.5 & $\approx 1$ & $<10^{-6}$ \\
    III & 1 &  23.3303 & 0.5 & $\approx 1$ & $0.06919 $ \\
    IV  & 0 &   0.1007 & 0.5 &      0.5252 & $0.57265 $ \\
    \hline
    IV  & 0 &   0.1007 & 0.4 &      0.4242 & $0.57265 $ \\
    IV  & 0 &   0.1007 & 0.3 &      0.3216 & $0.57265 $ \\
    IV  & 0 &   0.1007 & 0.2 &      0.2166 & $0.57265 $ \\
    IV  & 0 &   0.1007 & 0.1 &      0.1094 & $0.57265 $ \\
    \hline
  \end{tabular}
  \caption{Simulation truth of $\beta$ under various scenarios in second
  column. Third column contains the Bayes factor in favor of $\beta=1$.
  Fourth column contains the prior mean of $\beta$. Fifth column includes the
  posterior mean of $\beta$ under the prior for $\beta$ in the previous
  column. Sixth column includes the p-value of under the two-sample
  Kolmogorov-Smirnov test. In the first three scenarios, the posterior for
  $\beta$ in highly insensitive to the prior for $\beta$.}
  \label{tab:sim-results}
\end{table}

\section{Simulation Results}\label{sec:sim-results}
The following priors were used in this analysis. First, we set $K=6$ and
$p=0.5$. Then $\gamma_i\sim\Beta(1, 1)$, $\bm\eta_i\sim\Dirichlet_K(1/K)$,
$\mu_k\sim\Normal(\bar{\mu}, s_\mu^2)$, $\omega_k\sim\InvGamma(0.1, 0.1)$,
$\nu_k\sim\LogNormal(1.6, 0.4)$, $\psi_k\sim\Normal(-1, 1)$, where,
respectively, $\bar{\mu}$ and $s_\mu$ are the empirical mean and standard
deviation of the data for which $y_{i,n} > 0$.
%
We fit two models -- $\mathcal{M}_0$, with $\beta$ fixed at 0; and
$\mathcal{M}_1$, with $\beta$ fixed at 1. For each model, posterior inference
was made via Gibbs sampling. The initial 10000 MCMC samples were discarded as
burn-in, and every other subsequent sample in the next MCMC 10000 iterations
was kept for posterior inference. For each model, the inference speed was
approximately 32 iterations per second. We recovered the posterior probability
of $\mathcal{M}_1$ by computing the product of the Bayes factor and prior
odds in favor of $\mathcal{M}_1$. That is $\Pr(\beta=1 \mid \bm y) =
\text{BF}_{1,0} \cdot p/(1-p)$.
%
% In addition, updating the parameters $\bm\zeta, \bm v, \bm\mu, \bm\omega,
% \bm\nu$, and $\bm\psi$ multiple times for each update of the other parameters
% helped with mixing. Thus, 10 updates for those parameters were done during
% each iteration of the MCMC.
%
Table~\ref{tab:sim-results} contains the posterior inference for $\beta$
under the different scenarios. Included are the simulation truth of $\beta$,
the Bayes factor in favor of $\mathcal{M}_1$, the prior and posterior mean of
$\beta$, and p-value in an approximate two-sided Kolmogorov-Smirnov test. In
the first three scenarios, the posterior mean of $\beta$ matches the
simulation truth very well, and the prior mean of $\beta$ contributes little.
In scenario IV, a prior mean of 0.5 is insufficient to decisively conclude
whether the two distributions are different. Thus, we recommend setting
$p<0.5$ (e.g. $p=0.1$). When the two distributions are different, the prior mean
for $\beta$ does not influence the posterior substantially.
%
Figure~\ref{fig:sim-postdens-data-kde} summarizes the posterior predictive
densities for the positive values of $y_{i,n}$. Histograms of the observed
data are included, and the shaded curves are the 95\% credible intervals for
the densities. The posterior mean of the densities are included as solid
lines. Note that the intervals match the data closely.
% Also, in scenario III, where $\beta^\true=0$, the posterior mean of $\beta$
% was also 0. Thus, $\gamma_T$ and $\bm\eta_T$ were simply samples from the
% prior. Thus, the posterior predictive density for sample T was not included
% here.
%
We have also included Figure~\ref{fig:sim-postdens-data-true-den}, where the
solid lines represent the true density of the data. The graphs more clearly
show that the simulation truth is well captured by this model.
%
Figure~\ref{fig:sim-gammas} shows box plots of the posterior distribution of
$\gamma_i$ for the different scenarios. The circles represent the proportion
of zeros in the simulated data. The posterior distributions easily capture the
true values of $\gamma_i$.
%
\begin{figure}[t!]
  \centering
  \begin{tabular}{cc}
    \includegraphics[scale=.38]{results/K6/scenario1/img/postpred-data-hist.pdf} &
    \includegraphics[scale=.38]{results/K6/scenario2/img/postpred-data-hist.pdf} \\
    (a) Scenario I &
    (b) Scenario II \\
    \includegraphics[scale=.38]{results/K6/scenario3/img/postpred-data-hist.pdf} &
    \includegraphics[scale=.38]{results/K6/scenario4/img/postpred-data-hist.pdf} \\
    (c) Scenario III &
    (d) Scenario IV
  \end{tabular}
  \caption{Posterior predictive density in each simulation scenario for
  observed data ($y_{i,n}>0$), for $p=0.5$, overlaid with histogram of data.
  The shaded regions are 95\% credible intervals of the estimated density.}
  \label{fig:sim-postdens-data-kde}
\end{figure}

\begin{figure}[t!]
  \centering
  \begin{tabular}{cc}
    \includegraphics[scale=.45]{results/K6/scenario1/img/postpred-true-data-density.pdf} &
    \includegraphics[scale=.45]{results/K6/scenario2/img/postpred-true-data-density.pdf} \\
    (a) Scenario I &
    (b) Scenario II \\
    \includegraphics[scale=.45]{results/K6/scenario3/img/postpred-true-data-density.pdf} &
    \includegraphics[scale=.45]{results/K6/scenario4/img/postpred-true-data-density.pdf} \\
    (c) Scenario III &
    (d) Scenario IV
  \end{tabular}
  \caption{Posterior predictive density in each simulation scenario for
  observed data ($y_{i,n}>0$), for $p=0.5$. Solid lines are the true density
  of the simulated data. The shaded regions are 95\% credible intervals of
  the density.}
  \label{fig:sim-postdens-data-true-den}
\end{figure}

\begin{figure}[t!]
  \centering
  \begin{tabular}{cc}
    (a) Scenario I &
    (b) Scenario II \\
    \includegraphics[scale=.45]{results/K6/scenario1/img/gammas.pdf} &
    \includegraphics[scale=.45]{results/K6/scenario2/img/gammas.pdf} \\
    (c) Scenario III &
    (d) Scenario IV \\
    \includegraphics[scale=.45]{results/K6/scenario3/img/gammas.pdf} &
    \includegraphics[scale=.45]{results/K6/scenario4/img/gammas.pdf}
  \end{tabular}
  \caption{Box plots of posterior distribution of $\gamma_C$ and
  $\gamma_T^\star$ for different simulated datasets. Circles represent
  the proportion of zeros in each dataset. Blue for sample C, and red for
  sample T.}
  \label{fig:sim-gammas}
\end{figure}

% Uncomment if using bibliography:
% \bibliography{bib}
\end{document}