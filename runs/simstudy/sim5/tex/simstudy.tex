\documentclass[12pt]{article} % 12-point font

\usepackage[margin=1in]{geometry} % set page to 1-inch margins
\usepackage{bm,bbm} % for math
\usepackage{amsmath} % for math
\usepackage{amssymb} % like \Rightarrow
\setlength\parindent{0pt} % Suppresses the indentation of new paragraphs.

% Big display
\newcommand{\ds}{ \displaystyle }
% Parenthesis
\newcommand{\norm}[1]{\left\lVert#1\right\rVert}
\newcommand{\p}[1]{\left(#1\right)}
\newcommand{\bk}[1]{\left[#1\right]}
\newcommand{\bc}[1]{ \left\{#1\right\} }
\newcommand{\abs}[1]{ \left|#1\right| }
% Derivatives
\newcommand{\df}[2]{ \frac{d#1}{d#2} }
\newcommand{\ddf}[2]{ \frac{d^2#1}{d{#2}^2} }
\newcommand{\pd}[2]{ \frac{\partial#1}{\partial#2} }
\newcommand{\pdd}[2]{\frac{\partial^2#1}{\partial{#2}^2} }
% Distributions
\newcommand{\Normal}{\text{Normal}}
\newcommand{\Beta}{\text{Beta}}
\newcommand{\G}{\text{Gamma}}
\newcommand{\InvGamma}{\text{Inv-Gamma}}
\newcommand{\Uniform}{\text{Uniform}}
\newcommand{\Dirichlet}{\text{Dirichlet}}
\newcommand{\LogNormal}{\text{LogNormal}}
% Statistics
\newcommand{\E}{ \text{E} }
\newcommand{\iid}{\overset{iid}{\sim}}
\newcommand{\ind}{\overset{ind}{\sim}}
\newcommand{\true}{\text{TRUE}}

\usepackage{color}
\newcommand{\alert}[1]{\textcolor{red}{#1}}


% Graphics
\usepackage{graphicx}  % for figures
\usepackage{float} % Put figure exactly where I want [H]

% Uncomment if using bibliography
% Bibliography
% \usepackage{natbib}
% \bibliographystyle{plainnat}

% Adds settings for hyperlinks. (Mainly for table of contents.)
\usepackage{hyperref}
\hypersetup{
  pdfborder={0 0 0} % removes red box from links
}

% Title Settings
\title{Simulation Study 5}
\author{Arthur Lui}
\date{\today} % \date{} to set date to empty

% MAIN %
\begin{document}
\maketitle

\section{Simulation Setup}\label{sec:sim-setup}
We assessed our model through the following simulation studies. We first
generated four data sets (I, II, III, IV) according to our model. In each
four scenarios,
%
the true mixture locations were $\bm{\mu}^\true=(-1, 1, 3)$,
the true mixture scales were $\bm{\sigma}^\true=(0.7, 0.7, 0.7)$,
the true mixture degrees of freedom were $\bm{\nu}^\true=(7, 5, 10)$, and
the true mixture skews were $\bm{\phi}^\true=(-5, -3, 0)$.
%
In scenario I, $\gamma_C^\true=0.3$, $\gamma_T^\true=0.2$, $\bm\eta_C^\true=(0.5,
0.5, 0)$, and $\bm\eta_T^\true=(0.5,0.4,0.1)$. Implicitly, $\beta^\true=1$.
In scenario II, $\gamma_C^\true=0.3$, $\gamma_T^\true=0.3$, $\bm\eta_C^\true=(0.5,
0.5, 0)$, and $\bm\eta_T^\true=(0.5,0.4,0.1)$. Implicitly, $\beta^\true=1$.
In scenario III, $\gamma_C^\true=0.3$, $\gamma_T^\true=0.3$, $\bm\eta_C^\true=(0.5,
0.5, 0)$, and $\bm\eta_T^\true=(0.5,0.45,0.05)$. Implicitly, $\beta^\true=1$.
In scenario IV, $\gamma_C^\true=0.3$, $\gamma_T^\true=0.3$, $\bm\eta_C^\true=(0.5,
0.5, 0)$, and $\bm\eta_T^\true=(0.5,0.5,0)$. Implicitly, $\beta^\true=0$.
%
In each scenario, $N_i=1000$. Table~\ref{tab:sim-truth} summarizes the
simulation truth for the model parameters.
\begin{table}
  \centering
  \begin{tabular}{|c|ccccccc|}
    \hline 
    Scenario & $\gamma_C^\true$ & $\gamma_T^\true$ & $\bm\eta_C^\true$ & 
    $\bm\eta_T^\true$ & $\beta^\true$ & $\hat\beta$ & KS p-value \\
    \hline 
    I   & 0.3 & 0.2 & (0.5, 0.5, 0) & (0.5, 0.40, 0.10) & 1 & $\approx 1$ & $0.00003$ \\
    II  & 0.3 & 0.3 & (0.5, 0.5, 0) & (0.5, 0.40, 0.10) & 1 & $\approx 1$ & $0.01123$ \\
    III & 0.3 & 0.3 & (0.5, 0.5, 0) & (0.5, 0.45, 0.05) & 1 & $\approx 1$ & $0.28775$ \\
    IV  & 0.3 & 0.3 & (0.5, 0.5, 0) & (0.5, 0.50, 0.00) & 0 &      0.3921 & $0.64755$ \\
    \hline
  \end{tabular}
  \caption{Simulation truth under various scenarios. Posterior mean of
  $\beta$ is included before the right-most column. The right-most column
  is the p-value of under the two-sample Kolmogorov-Smirnov test.}
  \label{tab:sim-truth}
\end{table}

\section{Simulation Results}\label{sec:sim-results}
The following priors were used in this analysis. First, we set $K=5$ and
$p=0.5$. Then $\gamma_i\sim\Beta(1, 1)$, $\bm\eta_i\sim\Dirichlet_K(1/K)$,
$\mu_k\sim\Normal(\bar{\mu}, s_\mu^2)$, $\omega_k\sim\InvGamma(0.1, 0.1)$,
$\nu_k\sim\LogNormal(1.6, 0.4)$, $\psi_k\sim\Normal(-1, 1)$, where,
respectively, $\bar{\mu}$ and $s_\mu$ are the empirical mean and standard
deviation of the data for which $y_{i,n} > 0$.
%
\alert{
We fit two models -- $\mathcal{M}_0$, with $\beta$ fixed at 0; and
$\mathcal{M}_1$, with $\beta$ fixed at 1. For each model, posterior inference
was made via Gibbs sampling. The initial 3000 MCMC samples were discarded as
burn-in, and the subsequent 3000 samples were kept for posterior inference.
For each model, the inference speed was approximately 32 iterations per
second. We recover the posterior probability of $\mathcal{M}_1$ by computing
the product of the Bayes factor and prior odds in favor of $\mathcal{M}_1$.
That is $\Pr(\beta=1 \mid \bm y) = \text{BF}_{1,0} \cdot p/(1-p)$.
}
%
% In addition, updating the parameters $\bm\zeta, \bm v, \bm\mu, \bm\omega,
% \bm\nu$, and $\bm\psi$ multiple times for each update of the other parameters
% helped with mixing. Thus, 10 updates for those parameters were done during
% each iteration of the MCMC.
%
Figure~\ref{fig:sim-postdens-data-kde} summarizes the posterior predictive
densities for the positive values of $y_{i,n}$. The dashed lines are kernel
density estimates of the data, and the shaded regions are the 95\% credible
intervals for the densities. Note that the intervals match the data closely.
% Also, in scenario III, where $\beta^\true=0$, the posterior mean of $\beta$
% was also 0. Thus, $\gamma_T$ and $\bm\eta_T$ were simply samples from the
% prior. Thus, the posterior predictive density for sample T was not included
% here.
%
Since the KDE is only an approximation for the density of the observed data,
we have included Figure~\ref{fig:sim-postdens-data-true-den}, which replaces
the KDE of the observed portion of the simulated data with the actual pdf of
the data-generating mechanism. The graphs more clearly show that the simulation
truth is well captured by this model. 
%
Figure~\ref{fig:sim-gammas} shows box plots of the posterior distribution of
$\gamma_i$ for the different scenarios. The circles represent the proportion
of zeros in the simulated data. The posterior distributions easily capture the
true values of $\gamma_i$.
%
Note that Table~\ref{tab:sim-truth} also includes the posterior mean of
$\beta$, denoted $\hat\beta$. Note that when $\beta^\true=0$, $\hat\beta<
0.5$ as the Bayes factor marginalizes over model parameters and prefers
simpler models. By setting $p < 0.5$, to prefer simpler models
\emph{apriori}, the Bayes factor, and thus the posterior odds, in favor of
$\mathcal{M}_1$ decreases. For example, if $p=0.1$, then the $\Pr(\beta=1\mid
\bm y) \approx 0.06691$. As the distributions of the two samples become
increasingly different, $\hat\beta$ increases.
\begin{figure}[t!]
  \centering
  \begin{tabular}{cc}
    \includegraphics[scale=.45]{results/scenario1/img/postpred.pdf} &
    \includegraphics[scale=.45]{results/scenario2/img/postpred.pdf} \\
    (a) Scenario I &
    (b) Scenario II \\
    \includegraphics[scale=.45]{results/scenario3/img/postpred.pdf} &
    \includegraphics[scale=.45]{results/scenario4/img/postpred.pdf} \\
    (c) Scenario III &
    (d) Scenario IV
  \end{tabular}
  \caption{Posterior predictive density in each simulation scenario for
  observed data ($y_{i,n}>0$). Dashed lines are the kernel density estimates
  of the simulated data. The shaded regions are 95\% credible intervals of
  the density.}
  \label{fig:sim-postdens-data-kde}
\end{figure}

\begin{figure}[t!]
  \centering
  \begin{tabular}{cc}
    \includegraphics[scale=.45]{results/scenario1/img/postpred-true-data-density.pdf} &
    \includegraphics[scale=.45]{results/scenario2/img/postpred-true-data-density.pdf} \\
    (a) Scenario I &
    (b) Scenario II \\
    \includegraphics[scale=.45]{results/scenario3/img/postpred-true-data-density.pdf} &
    \includegraphics[scale=.45]{results/scenario4/img/postpred-true-data-density.pdf} \\
    (c) Scenario III &
    (d) Scenario IV
  \end{tabular}
  \caption{Posterior predictive density in each simulation scenario for
  observed data ($y_{i,n}>0$). Dashed lines are the kernel density estimates
  of the simulated data. The shaded regions are 95\% credible intervals of
  the density.}
  \label{fig:sim-postdens-data-true-den}
\end{figure}

\begin{figure}[t!]
  \centering
  \begin{tabular}{cc}
    (a) Scenario I &
    (b) Scenario II \\
    \includegraphics[scale=.45]{results/scenario1/img/gammas.pdf} &
    \includegraphics[scale=.45]{results/scenario2/img/gammas.pdf} \\
    (c) Scenario III &
    (d) Scenario IV \\
    \includegraphics[scale=.45]{results/scenario3/img/gammas.pdf} &
    \includegraphics[scale=.45]{results/scenario4/img/gammas.pdf}
  \end{tabular}
  \caption{Box plots of posterior distribution of $\gamma_C$ and
  $\gamma_T^\star$ for different simulated datasets. Circles represent
  the proportion of zeros in each dataset. Blue for sample C, and red for
  sample T.}
  \label{fig:sim-gammas}
\end{figure}

% Uncomment if using bibliography:
% \bibliography{bib}
\end{document}
