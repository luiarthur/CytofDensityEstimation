\documentclass[12pt]{article} % 12-point font

\usepackage[margin=1in]{geometry} % set page to 1-inch margins
\usepackage{bm,bbm} % for math
\usepackage{amsmath} % for math
\usepackage{amssymb} % like \Rightarrow
\setlength\parindent{0pt} % Suppresses the indentation of new paragraphs.

% Big display
\newcommand{\ds}{ \displaystyle }
% Parenthesis
\newcommand{\norm}[1]{\left\lVert#1\right\rVert}
\newcommand{\p}[1]{\left(#1\right)}
\newcommand{\bk}[1]{\left[#1\right]}
\newcommand{\bc}[1]{ \left\{#1\right\} }
\newcommand{\abs}[1]{ \left|#1\right| }
% Derivatives
\newcommand{\df}[2]{ \frac{d#1}{d#2} }
\newcommand{\ddf}[2]{ \frac{d^2#1}{d{#2}^2} }
\newcommand{\pd}[2]{ \frac{\partial#1}{\partial#2} }
\newcommand{\pdd}[2]{\frac{\partial^2#1}{\partial{#2}^2} }
% Distributions
\newcommand{\Normal}{\text{Normal}}
\newcommand{\Beta}{\text{Beta}}
\newcommand{\G}{\text{Gamma}}
\newcommand{\InvGamma}{\text{Inv-Gamma}}
\newcommand{\Uniform}{\text{Uniform}}
\newcommand{\Dirichlet}{\text{Dirichlet}}
\newcommand{\LogNormal}{\text{LogNormal}}
% Statistics
\newcommand{\E}{ \text{E} }
\newcommand{\iid}{\overset{iid}{\sim}}
\newcommand{\ind}{\overset{ind}{\sim}}
\newcommand{\true}{\text{TRUE}}

\usepackage{color}
\newcommand{\alert}[1]{\textcolor{red}{#1}}


% Graphics
\usepackage{graphicx}  % for figures
\usepackage{float} % Put figure exactly where I want [H]

% Uncomment if using bibliography
% Bibliography
% \usepackage{natbib}
% \bibliographystyle{plainnat}

% Adds settings for hyperlinks. (Mainly for table of contents.)
\usepackage{hyperref}
\hypersetup{
  pdfborder={0 0 0} % removes red box from links
}

% Title Settings
\title{Simulation Study 1}
\author{Arthur Lui}
\date{\today} % \date{} to set date to empty

% MAIN %
\begin{document}
\maketitle

\section{Simulation Setup}\label{sec:sim-setup}
We assessed our model through the following simulation studies. We first
generated three data sets (I, II, III) according to our model. In each three
scenarios, 
%
the true mixture locations were $\bm{\mu}^\true=(-1, 1, 3)$,
the true mixture scales were $\bm{\sigma}^\true=(0.7, 0.7, 0.7)$,
the true mixture degrees of freedom were $\bm{\nu}^\true=(15, 30, 10)$, and
the true mixture skews were $\bm{\phi}^\true=(-20, -5, 0)$.
%
In scenario I, $\gamma_C^\true=0.3$, $\gamma_T^\true=0.2$, $\bm\eta_C^\true=(0.5,
0.5, 0)$, and $\bm\eta_T^\true=(0.5,0.2,0.3)$. Implicitly, $\beta^\true=1$.
In scenario II, $\gamma_C^\true=0.3$, $\gamma_T^\true=0.3$, $\bm\eta_C^\true=(0.5,
0.5, 0)$, and $\bm\eta_T^\true=(0.5,0.2,0.3)$. Implicitly, $\beta^\true=1$.
In scenario III, $\gamma_C^\true=\gamma_T^\true=0.3$, and
$\bm\eta_C^\true=\bm\eta_T^\true=(0.5, 0.5, 0)$. Implicitly, $\beta^\true=0$.
%
In each scenario, $N_i=1000$. Table~\ref{tab:sim-truth} summarizes the
simulation truth for the model parameters.
\begin{table}
  \centering
  \begin{tabular}{|c|cccccc|}
    \hline 
    Scenario & $\gamma_C^\true$ & $\gamma_T^\true$ & $\bm\eta_C^\true$ & 
    $\bm\eta_T^\true$ & $\beta^\true$ & $\hat\beta$ \\
    \hline 
    I   & 0.3 & 0.2 & (0.5, 0.5, 0) & (0.5, 0.2, 0.3) & 1 & 0.968 \\
    II  & 0.3 & 0.3 & (0.5, 0.5, 0) & (0.5, 0.2, 0.3) & 1 & 1.000 \\
    III & 0.3 & 0.3 & (0.5, 0.5, 0) & (0.5, 0.5, 0.0) & 0 & 0.001 \\
    \hline
  \end{tabular}
  \caption{Simulation truth under various scenarios. Posterior mean of
  $\beta$ is included in the right-most column.}
  \label{tab:sim-truth}
\end{table}

\section{Simulation Results}\label{sec:sim-results}
The following priors were used in this analysis.
First, we set $K=6$. Then $p\sim\Beta(.1, .9)$, $\gamma_i\sim\Beta(0.5,
0.5)$, $\bm\eta_i\sim\Dirichlet_K(1/K)$, $\mu_k\sim\Normal(\bar{\mu}, s_\mu^2)$,
$\omega_k\sim\InvGamma(3, 2)$, $\nu_k\sim\LogNormal(4, 0.01)$,
$\psi_k\sim\Normal(0, 3^2)$, where, respectively, $\bar{\mu}$ and $s_\mu$ are
the empirical mean and standard deviation of the data for which $y_{i,n} >
0$. Posterior inference was made via Gibbs sampling. The initial 1000 MCMC
samples were discarded as burn-in, and the subsequent 1000 samples were kept
for analysis. The inference speed was approximately 45 iterations per second.
Figure~\ref{fig:sim-postdens} summarizes the posterior densities for the
positive values of $y_{i,n}$. The dashed lines are kernel density estimates
of the data, and the shaded regions are the 95\% credible intervals for the
densities. Note that the intervals match the data closely. Also, in scenario
III, where $\beta^true=0$, the posterior mean of $\beta$ was also 0. Thus,
$\gamma_T$ and $\bm\eta_T$ were simply samples from the prior. Thus, the 
posterior density for sample T was not included here.

\alert{Juhee: There seems to be difficulty in estimating $\nu_k$, $v_{i,n}$,
and $\zeta_{i,n}$. $\nu_k$ and $v_{i,n}$ related to the degrees of freedom
and $\zeta$ and $\psi_k$ relate to the skewness. I also wasn't able to obtain
decent results unless I used the skew-$t$ pdf in the mixture in updating
$\beta$, and in updating $\lambda_{i,n}$. Changing $K$ does affect the
inference. In particular, when I used a $K=3$ I did not always recover the
simulation truth, and at times the fit was poor. Using a larger $K$ did not
always yield the original $\bm\eta_i$, but the fit was usually good.
Under the same setup but with $\sigma_i^\true=0.1$ in the simulation truth,
the credible intervals were much narrower.}.

\begin{figure}[t!]
  \centering
  \begin{tabular}{c}
    \includegraphics[scale=.45]{results/scenario1/img/postpred.pdf} \\
    (a) Scenario I \\
    \includegraphics[scale=.45]{results/scenario2/img/postpred.pdf} \\
    (b) Scenario II \\
    \includegraphics[scale=.45]{results/scenario3/img/postpred.pdf} \\
    (c) Scenario III \\
  \end{tabular}
  \caption{Posterior density in each simulation scenario for observed data
  ($y_{i,n}>0$). Dashed lines are the kernel density estimates of the
  simulated data. The shaded regions are 95\% credible intervals of the
  density. \alert{Images need to be re-formatted.}}
  \label{fig:sim-postdens}
\end{figure}

\alert{TODO: Discuss Kolmogorov Smirnov statistic...}

% Uncomment if using bibliography:
% \bibliography{bib}
\end{document}