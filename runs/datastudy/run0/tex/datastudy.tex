\documentclass[12pt]{article} % 12-point font

\usepackage[margin=1in]{geometry} % set page to 1-inch margins
\usepackage{bm,bbm} % for math
\usepackage{amsmath} % for math
\usepackage{amssymb} % like \Rightarrow
\setlength\parindent{0pt} % Suppresses the indentation of new paragraphs.

% Big display
\newcommand{\ds}{ \displaystyle }
% Parenthesis
\newcommand{\norm}[1]{\left\lVert#1\right\rVert}
\newcommand{\p}[1]{\left(#1\right)}
\newcommand{\bk}[1]{\left[#1\right]}
\newcommand{\bc}[1]{ \left\{#1\right\} }
\newcommand{\abs}[1]{ \left|#1\right| }
% Derivatives
\newcommand{\df}[2]{ \frac{d#1}{d#2} }
\newcommand{\ddf}[2]{ \frac{d^2#1}{d{#2}^2} }
\newcommand{\pd}[2]{ \frac{\partial#1}{\partial#2} }
\newcommand{\pdd}[2]{\frac{\partial^2#1}{\partial{#2}^2} }
% Distributions
\newcommand{\Normal}{\text{Normal}}
\newcommand{\Beta}{\text{Beta}}
\newcommand{\G}{\text{Gamma}}
\newcommand{\InvGamma}{\text{Inv-Gamma}}
\newcommand{\Uniform}{\text{Uniform}}
\newcommand{\Dirichlet}{\text{Dirichlet}}
\newcommand{\LogNormal}{\text{LogNormal}}
% Statistics
\newcommand{\E}{ \text{E} }
\newcommand{\iid}{\overset{iid}{\sim}}
\newcommand{\ind}{\overset{ind}{\sim}}
\newcommand{\true}{\text{TRUE}}

\usepackage{color}
\newcommand{\alert}[1]{\textcolor{red}{#1}}


% Graphics
\usepackage{graphicx}  % for figures
\usepackage{float} % Put figure exactly where I want [H]

% Uncomment if using bibliography
% Bibliography
% \usepackage{natbib}
% \bibliographystyle{plainnat}

% Adds settings for hyperlinks. (Mainly for table of contents.)
\usepackage{hyperref}
\hypersetup{
  pdfborder={0 0 0} % removes red box from links
}

% Title Settings
\title{CyTOF Density Estimation Data Analysis}
\author{Arthur Lui}
\date{\today} % \date{} to set date to empty

% MAIN %
\begin{document}
\maketitle

\section{Preliminary Data Analysis}
The markers CD3z, EOMES, Perforin, and Siglec7 for one subject were studied
in this data analysis. The number of cells in each sample are summarized in 
Table~\ref{tab:data-counts}.
%
\alert{Juhee: This data set was randomly subsampled such that only 20000
cells total are used. That is $N_C + N_T = 20000$.}
%
We set $K=5$. Beyond that, the same procedure for determining the priors for
this analysis was the same as that in the simulation studies. Posterior
samples were obtained via MCMC as previously outlined. The initial 2000
samples were discarded as burn, and the subsequent 10000 samples were kept
for analysis. Figures~\ref{fig:data-post-pred}~and~\ref{fig:data-post-gamma}
show the posterior density and posterior distribution for $\gamma_i$ for each
marker. Also included is the posterior mean for $\beta$, denoted by
$\hat\beta$. Note that for marker Siglec7, $\hat\beta=0$; and $\hat\beta=1$
for all other markers. Overall, when $\beta=1$ the model fit is excellent and
the posterior densities contain the kernel density estimates and follow them
closely.

\alert{Juhee: It seems there's a problem when $\beta=0$. For Siglec7,
$\hat\beta$ is estimated to be 0. This somewhat makes sense from
Figure~\ref{fig:data-post-pred} as the two KDEs are very similar. But in
Figure~\ref{fig:data-post-gamma}~(d), $\gamma_C$ is combining both samples,
but neither $\gamma_C$ nor $\gamma_T^\star$ (which in this case are the same)
match the data well. I think part of the issue is that when $\beta=0$,
$\gamma_T$ and $\eta_T$ are simply samples from the prior $\Beta(1,1)$. The
chance of drawing good $\gamma_T$ and $\eta_T$ from the priors is small, so
$\beta$ gets stuck for a long time. In essence, even though $Z_i/N_i$ is
small for $i\in\bc{C,T}$ I think $\hat\beta$ should be non-zero, and the MCMC
is simply stuck. What are your thoughts?}

\begin{table}[!t]
  \centering
  \begin{tabular}{|c|rrrr|}
    \hline
    Marker   & $N_C$ & $N_T$ & $Z_C$ & $Z_T$ \\ 
    \hline
    CD3z       & 9730 & 10270 & 159 &  47 \\ 
    EOMES      & 9730 & 10270 & 939 & 806 \\ 
    Perforin   & 9730 & 10270 &  10 & 102 \\ 
    Siglec7    & 9730 & 10270 & 610 & 465 \\ 
    % Granzyme A & 9730 & 10270 &   4 &  12 \\ 
    \hline
  \end{tabular} 
  \caption{Counts of the number of cells in donor sample before ($N_C$) and
  after ($N_T$) treatment. $Z_C$ and $Z_T$ respectively denote the proportion
  of zeros in the measurements before and after treatment.}
  \label{tab:data-counts}
\end{table}

\begin{figure}[t!]
  \centering
  \begin{tabular}{cc}
    \includegraphics[scale=0.5]{results/donor1/CD3z/img/postpred.pdf} &
    \includegraphics[scale=0.5]{results/donor1/EOMES/img/postpred.pdf} \\
    (a) CD3z ($\hat\beta=1$) & (b) EOMES ($\hat\beta=1$) \\
    %
    \includegraphics[scale=0.5]{results/donor1/Perforin/img/postpred.pdf} &
    \includegraphics[scale=0.5]{results/donor1/Siglec7/img/postpred.pdf} \\
    % \includegraphics[scale=0.5]{results/donor1/Granzyme_A/img/postpred.pdf} \\
    (c) Perforin ($\hat\beta=1$) & (d) Siglec7 ($\hat\beta=0$) \\
  \end{tabular}
  \caption{The dashed blue and red lines are, respectively, the kernel
  density estimates of $\bm{\tilde{y}}_C$ and $\bm{\tilde{y}}_T$. Likewise,
  the blue and red shaded regions are the posterior density estimates for
  $\bm{\tilde{y}}_C$ and $\bm{\tilde{y}}_T$, respectively.}
  \label{fig:data-post-pred}
\end{figure}

\begin{figure}[t!]
  \centering
  \begin{tabular}{cc}
    \includegraphics[scale=0.5]{results/donor1/CD3z/img/gammas.pdf} &
    \includegraphics[scale=0.5]{results/donor1/EOMES/img/gammas.pdf} \\
    (a) CD3z ($\hat\beta=1$) & (b) EOMES ($\hat\beta=1$) \\
    %
    \includegraphics[scale=0.5]{results/donor1/Perforin/img/gammas.pdf} &
    \includegraphics[scale=0.5]{results/donor1/Siglec7/img/gammas.pdf} \\
    % \includegraphics[scale=0.5]{results/donor1/Granzyme_A/img/gammas.pdf} \\
    (c) Perforin ($\hat\beta=1$) & (d) Siglec7 ($\hat\beta=0$) \\
  \end{tabular}
  \caption{Posterior distributions of $\gamma_C$ and $\gamma_T^\star$ in blue
  and red respectively. The circles represent the proportion of zeros in each
  sample.}
  \label{fig:data-post-gamma}
\end{figure}


% Uncomment if using bibliography:
% \bibliography{bib}
\end{document}